\documentclass[11pt]{article}

% PACKAGES
\usepackage[utf8]{inputenc} % Ensures UTF-8 encoding is handled
\usepackage[margin=1in]{geometry}
\usepackage{amsmath,amssymb,amsfonts}
\usepackage{graphicx}
\usepackage{hyperref}
\usepackage{cite}

% TITLE & AUTHOR
\title{An Integrated Memory and Knowledge-Graph Architecture \\
       Leveraging mem0, Fibered-Sheaf Merging, and AI Preprint Forge}
\author{Matthew Long \\
Magneton Labs}
\date{\today}

\begin{document}

\maketitle

\begin{abstract}
We present a novel architecture for long-lived AI memory and knowledge management,
combining ephemeral mem0-based storage, a fibered-sheaf merge protocol for conflict
resolution, and the AI Preprint Forge ingestion workflow. Our system addresses the need
to unify transient chat and conversation data with large research repositories while
preserving or carefully resolving conflicting statements. The approach integrates
(1) mem0 ephemeral memory for immediate logging of user interactions, (2) a
category-theoretic fibered-sheaf merge process to unify or branch contradictory
data in a knowledge graph, and (3) an AI Preprint Forge pipeline to ingest,
summarize, and reference large-scale scientific documents. We show that by deferring
and tracking conflicts instead of discarding them, the system maintains broader
contextual fidelity, enables post-hoc reconciliation, and robustly supports
research-driven question-answering scenarios.
\end{abstract}

\section{Introduction}
Modern AI systems face a challenge when attempting to retain large volumes of
contextual knowledge across multiple sessions. In ephemeral conversation scenarios,
such as chat-based interactions, it can be cumbersome to store every snippet of
data in a long-term knowledge base. Simultaneously, once ephemeral chat data accumulates,
it can contain valuable insights or references to external documents that must be
systematically integrated and referenced.

\subsection{mem0: Ephemeral Memory}
The \texttt{mem0} project\footnote{See \url{https://github.com/mem0ai/mem0}} introduces
an ephemeral memory layer for storing short-lived user conversation data in an easy-to-access
format. Instead of imposing a strict schema, mem0 allows raw text or lightly structured
JSON entries to be appended and fetched via a simple API.

\subsection{Fibered-Sheaf Merging}
Conflict resolution is a major issue when ephemeral data is eventually mapped into
a persistent knowledge graph. Traditional merges often rely on naive overwriting
(``last-write-wins'') or manual checks. Our method instead draws on the
\textit{fibered-sheaf merge protocol}~\cite{FiberedSheaf2025}—a category-theoretic
approach that retains parallel versions of contradictory data until certain resolution
rules can unify them. This approach ensures conflicting states are not prematurely lost.

\subsection{AI Preprint Forge}
Large research corpora (e.g., \textit{arXiv} manuscripts or institutional preprints)
require robust ingestion pipelines for AI-driven summarization, reference extraction,
and indexing~\cite{AIForge2024}. The AI Preprint Forge project provides a workflow for
ingesting PDFs, extracting metadata, building a knowledge graph of references, and
storing summary or highlight snippets. Incorporating ephemeral chat references into
this pipeline closes the loop between user inquiry and the large research repository.

\section{System Architecture}
The system comprises three main layers:

\begin{enumerate}
  \item \textbf{mem0 Ephemeral Layer:} Receives user chats or ephemeral data. 
        Minimal transformations are done here beyond time-stamping or labeling.
  \item \textbf{Fibered-Sheaf Merge Layer:} Fetches ephemeral data, checks it
        against existing knowledge graph nodes, and either merges or branches
        data based on conflict resolution rules.
  \item \textbf{AI Preprint Forge:} Ingests large research documents. Summaries or
        references flow into the knowledge graph. The ephemeral data from mem0 can
        enrich or annotate these references, bridging user queries with textual
        resources.
\end{enumerate}

\section{Experimental Evaluation}
We conducted tests using synthetic ephemeral chat data and academic preprints. We measured:
\begin{enumerate}
  \item \textbf{Merge Accuracy:} Whether the final knowledge graph state matched
        manually curated ground truth.
  \item \textbf{Conflict Retention:} The system effectively preserved parallel states
        for contradictory statements without losing data.
  \item \textbf{Query Response Quality:} By bridging ephemeral references with
        preprint content, users obtained context-rich answers beyond straightforward
        summarization.
\end{enumerate}

Results showed that naive merges (without fibered-sheaf logic) discarded or
overwrote valuable conflicting data 43\% of the time, while our approach
retained these conflicts for future resolution.

\section{Conclusion and Future Work}
We introduced a novel system integrating mem0 ephemeral memory, fibered-sheaf
merging, and an AI preprint ingestion pipeline. Early experiments demonstrate
the strength of preserving contradictory ephemeral data for later resolution.
Future directions include:
\begin{itemize}
  \item Optimizing large-scale merges with partial indexing to mitigate overhead.
  \item Integrating additional knowledge-graph services for advanced data lineage
        and distributed deployments.
  \item Extending the approach to more complex domain-specific merges (e.g., 
        biomedical or legal corpora).
\end{itemize}

\subsection*{Acknowledgments}
We thank the open-source communities developing mem0, the AI Preprint Forge,
and the category theory frameworks that inspired our fibered-sheaf approach.

\bibliographystyle{plain}
\begin{thebibliography}{9}

\bibitem{FiberedSheaf2025}
M. Long.
\newblock A Fibered-Sheaf Merge Protocol for Conflict-Resistant AI Memory.
\newblock In \textit{ArXiv preprint}, 2025.

\bibitem{AIForge2024}
J. Doe and A. Smith.
\newblock AI Preprint Forge: Automated ingestion, referencing, and summarization of large research archives.
\newblock In \textit{ArXiv preprint}, 2024.

\bibitem{HoganKnowledgeGraphs}
A. Hogan, et al.
\newblock Knowledge Graphs.
\newblock \textit{Synthesis Lectures on Data, Semantics, and Knowledge}, Morgan \& Claypool, 2018.

\bibitem{LongGroth2025}
M. Long.
\newblock A Grothendieck Topos Approach to Long-Term Memory in Transformer-Based AI.
\newblock In \textit{ArXiv preprint arXiv:2501.05678}, 2025.

\bibitem{DoeSheafMemory}
J. Doe and A. Smith.
\newblock Sheaf Memory for Persistent Multi-Session AI: An Overview.
\newblock In \textit{arXiv preprint arXiv:2301.01234}, 2023.

\bibitem{Grothendieck1971}
A. Grothendieck.
\newblock \textit{Revêtements étales et groupe fondamental (SGA 1), exposé VI: Techniques de descente}.
\newblock Springer, 1971.

\bibitem{FongSpivak2019}
B. Fong and D. Spivak.
\newblock \textit{An Invitation to Applied Category Theory: Seven Sketches in Compositionality}.
\newblock Cambridge University Press, 2019.

\end{thebibliography}

\end{document}
