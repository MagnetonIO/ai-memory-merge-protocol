\documentclass[12pt]{article}

\usepackage[margin=1in]{geometry}
\usepackage{setspace}
\usepackage{hyperref}
\setstretch{1.2}

\title{Patent Application: \\
       Systems and Methods for Integrating \\
       Ephemeral Memory, Fibered-Sheaf Merge, \\
       and Research Ingestion Pipelines}
\author{Inventor: Matthew Long \\
        Assignee: Magneton Labs}
\date{\today}

\begin{document}
\maketitle

\begin{center}
{\Large \textbf{PATENT APPLICATION}}
\end{center}

\section*{CROSS-REFERENCE TO RELATED APPLICATIONS}
This application claims priority to and the benefit of any relevant provisional 
applications (if applicable). Reference is made to concurrently filed or 
previously filed applications, including (List them if any).

\section*{FIELD OF THE INVENTION}
The present invention relates generally to systems and methods of knowledge management 
in artificial intelligence, and more particularly, to integrating ephemeral memory 
storage with a fibered-sheaf merging approach, combined with an automated ingestion 
pipeline for research documents.

\section*{BACKGROUND}
Modern AI requires robust memory architectures to store both transient user interaction data 
and stable, high-volume knowledge from research papers, technical documents, or user-provided 
content. Conventional approaches either (i) discard ephemeral data after a short period or 
(ii) merge ephemeral data into a knowledge base using simplistic conflict resolution. These 
limitations cause data loss and hamper accurate knowledge representation.

Attempts to integrate ephemeral chat logs with knowledge graphs often face contradictions 
(e.g., multiple versions of a fact). Existing solutions typically overwrite conflicting versions, 
resulting in possible misrepresentations. Moreover, ingesting large-scale research collections 
into the same knowledge store introduces further complexities in reconciling ephemeral user statements 
with authoritative external documents.

\section*{SUMMARY OF THE INVENTION}
Embodiments of the present invention provide a novel system and method that:
\begin{enumerate}
  \item Captures ephemeral (short-term) memory using a lightweight data layer (e.g., mem0).
  \item Merges ephemeral data into a persistent knowledge graph via a fibered-sheaf protocol,
        preserving parallel states for conflicting information.
  \item Integrates an automated ingestion workflow (AI Preprint Forge) that processes 
        research documents, extracting references, metadata, and text, thus unifying ephemeral 
        user queries with stable academic sources.
\end{enumerate}

Through the fibered-sheaf methodology, contradictory data is not immediately discarded or 
overwritten. Instead, the system either resolves conflicts via preconfigured rules (confidence, 
timestamp, user override) or preserves them as separate branches to be reconciled if and 
when further evidence emerges. 

\section*{BRIEF DESCRIPTION OF THE DRAWINGS}
No formal drawings are provided in this example, but one skilled in the art may
prepare block diagrams illustrating:
\begin{enumerate}
  \item The mem0 ephemeral memory subsystem,
  \item The fibered-sheaf merge engine,
  \item The AI Preprint Forge ingestion pipeline,
  \item Communication between these components and the knowledge graph database.
\end{enumerate}

\section*{DETAILED DESCRIPTION OF THE INVENTION}

\subsection*{System Overview}
An embodiment includes three major components:
\begin{itemize}
  \item \textbf{Ephemeral Memory Layer (mem0):} Stores chat logs or short-lived data
        in a raw JSON format with minimal indexing, allowing quick write/read operations.
  \item \textbf{Fibered-Sheaf Merge Engine:} Periodically or on-demand, this engine
        fetches ephemeral entries, maps them to relevant nodes in the knowledge graph,
        checks for data consistency, and merges or branches data accordingly.
  \item \textbf{Research Ingestion (AI Preprint Forge):} Processes large documents (PDFs,
        text, or otherwise) to extract structured metadata. Summaries or references are 
        then used in the knowledge graph, enabling cross-referencing with ephemeral chat logs.
\end{itemize}

\subsection*{Fibered-Sheaf Conflict Resolution}
The invention leverages a “fibered-sheaf” data structure, wherein each knowledge node can
include one or more possible states (fibers). When ephemeral data conflicts with existing
states:
\begin{itemize}
  \item Automatic rules (e.g., time-based, confidence-based) may unify one version.
  \item If no rule resolves the conflict, the system retains multiple states in separate
        fibers, which remain accessible for future queries or merges.
\end{itemize}

\subsection*{API and Data Flow}
The invention includes an API that exposes:
\begin{enumerate}
  \item \texttt{POST /mem0/entries} - store ephemeral data,
  \item \texttt{POST /fs-merge} - initiate or schedule a fibered-sheaf merge,
  \item \texttt{POST /preprint-forge/ingest} - ingest a research document for analysis.
\end{enumerate}
Data then flows into the knowledge graph, updated in real-time or batch processes.

\subsection*{Advantages and Use Cases}
\begin{itemize}
  \item \textbf{Preservation of Conflicting Data:} Ensures alternative versions are kept 
        until more information is available, improving reliability.
  \item \textbf{Context-Aware Research Assistance:} Merging ephemeral queries with 
        preprint data helps produce context-rich answers and fosters better Q\&A experiences.
  \item \textbf{Scalability:} The layered approach allows ephemeral data to be gradually 
        distilled or summarized and then integrated with large knowledge corpora.
\end{itemize}

\section*{CLAIMS}
\begin{enumerate}
\item \textbf{A system for integrating ephemeral memory with a persistent knowledge graph,}
comprising:
\begin{enumerate}
  \item an ephemeral memory layer configured to store incoming user data in a short-term format,
  \item a fibered-sheaf merge engine configured to retrieve said short-term data and check 
        for conflicts with existing nodes in a knowledge graph, and
  \item a resolution module configured to unify or branch conflicting data states.
\end{enumerate}

\item \textbf{The system of claim 1,} wherein the merge engine applies confidence scores or
timestamps as part of an automatic resolution method.

\item \textbf{The system of claim 1,} further comprising an ingestion subsystem configured
to parse research documents and add derived metadata into the knowledge graph.

\item \textbf{The system of claim 1,} wherein if conflicting data cannot be resolved automatically,
parallel states are preserved in the knowledge graph for future reconciliation.

\item \textbf{A method of operating the system of claim 1,} comprising:
\begin{enumerate}
  \item receiving ephemeral data from a user session,
  \item identifying one or more corresponding nodes in the knowledge graph,
  \item determining whether data conflicts exist between ephemeral data and existing node states,
  \item applying a fibered-sheaf merge procedure to automatically unify or branch data, 
        thereby ensuring no loss of conflicting information, and
  \item exposing an API allowing external systems or users to query or update the knowledge graph 
        with ephemeral data or external research content.
\end{enumerate}
\end{enumerate}

\section*{ABSTRACT}
A system and method integrate ephemeral data stored in a lightweight memory layer with a
persistent knowledge graph using a fibered-sheaf protocol. Conflicting data states are
either resolved or preserved for future resolution, ensuring minimal loss of information.
An ingestion pipeline automatically processes large research documents to update the graph
with stable, high-confidence data. The invention improves AI memory fidelity and
enhances research-based user interactions.

\section*{CONCLUSION}
The above description details various embodiments of the invention. Variations
and modifications can be made by one skilled in the art without departing from
the scope and spirit of the invention, as set forth in the appended claims.

\end{document}
